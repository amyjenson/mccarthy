% Copyright 2009--2022 Ed Bueler

\documentclass[10pt,hyperref={pdfpagelabels=true}]{beamer}

\usetheme{Boadilla}

\usepackage{times}
\usepackage{hyperref}
\usepackage[T1]{fontenc}
\usepackage{verbatim}
\usepackage{empheq}
\usepackage{color}
\usepackage{xmpmulti}  % more portable than animate
\usepackage{graphicx}
\usepackage{tikz}

\newcommand{\ddt}[1]{\ensuremath{\frac{\partial #1}{\partial t}}}
\newcommand{\ddx}[1]{\ensuremath{\frac{\partial #1}{\partial x}}}
\newcommand{\ddy}[1]{\ensuremath{\frac{\partial #1}{\partial y}}}
\newcommand{\pp}[2]{\ensuremath{\frac{\partial #1}{\partial #2}}}
\renewcommand{\t}[1]{\texttt{#1}}
\newcommand{\Matlab}{\textsc{Matlab}\xspace}
\newcommand{\bq}{\mathbf{q}}
\newcommand{\bU}{\mathbf{U}}
\newcommand{\eps}{\epsilon}
\newcommand{\grad}{\nabla}
\newcommand{\Div}{\nabla\cdot}
\newcommand{\devstress}{\tau}

\newcommand{\mmess}[1]{\vspace{-0.1in}\begin{center}
\fbox{\url{#1.m}}
\end{center}}

\newcommand{\minput}[1]{\scriptsize\verbatiminput{tmp/#1.slim.m}
\tiny\mmess{#1}\normalsize}

\newcommand{\minputtiny}[1]{\tiny\verbatiminput{tmp/#1.slim.m}
\mmess{#1}\normalsize}



\title[Numerical modelling]{Numerical modelling of \\ glaciers, ice sheets, and ice shelves}

\author[Bueler]{Ed Bueler}

\institute[UAF]
{
  Dept of Math \& Stat, University of Alaska Fairbanks \\
  \texttt{elbueler@alaska.edu}
}

\date[]{{\scriptsize McCarthy, Alaska: 2010, 2012, 2014, 2016, 2018, 2022}}

%\titlegraphic{\vspace{-20mm} \includegraphics[width=0.7\textwidth]{roughfinal}}

\beamertemplatenavigationsymbolsempty

\begin{document}
\graphicspath{{../figures/}}

{
% centered background image
\usebackgroundtemplate{%
\parbox[c][\paperheight][b]{\paperwidth}{\centering \tikz\node[opacity=0.2] {\includegraphics[width=0.8\paperwidth]{whiteoutroughfinal}};}
}

\begin{frame}
  \maketitle
\end{frame}
}

%\AtBeginSection[]{}

\AtBeginSection[]
{
  \begin{frame}<beamer>
    \frametitle{Outline}
    \tableofcontents[currentsection,hideallsubsections]
  \end{frame}
}


\begin{frame}{slogans \& scope}

slogans: I will
  \begin{itemize}
  \item \alert{approximate ice flow}
  \item \alert{show codes that actually work}
  \item \alert{always focus on the continuum model}
  \end{itemize}
\medskip

scope:
  \begin{itemize}
  \item models

    \begin{itemize}
    \item[$\circ$] shallow ice approximation (SIA) in 2D
    \item[$\circ$] shallow shelf approximation (SSA) in 1D \only<2>{\alert{X}}
    \item[$\circ$] mass continuity \& surface kinematical equations
    \end{itemize}

  \item numerical ideas

    \begin{itemize}
    \item[$\circ$] finite difference schemes
    \item[$\circ$] solving algebraic systems from stress balances
    \item[$\circ$] verification
    \end{itemize}
  \end{itemize}
\end{frame}


\begin{comment}
\begin{frame}{outside of scope}

\large\emph{not} \normalsize covered here:\normalsize
\medskip

  \begin{itemize}
  \item Stokes and ``higher order'' flow equations
  \item thermomechanical coupling or polythermal ice
  \item subglacial hydrology/processes
  \item mass balance and snow/firn processes
  \item constitutive relations other than Glen isotropic
  \item grounding lines, calving fronts, ocean interaction
  \item paleo-climate and ``spin-up''
  \item earth deformation under ice sheet load
  \item other numerics: FEM, spectral, multigrid, parallel, \dots
  \item etc.
  \end{itemize}

\end{frame}
\end{comment}

\begin{frame}{notation} 

\begin{center}
  \includegraphics[width=0.9\textwidth]{flowline}

\tiny \emph{figure modified from} Schoof (2007)
\end{center}

\scriptsize
  \begin{itemize}
  \item coordinates $t,x,y,z$  (with $z$ vertical, positive upward)
  \item subscripts for partial derivatives $u_x = \partial u/\partial x$
  \item $H=$ ice thickness
  \item $h=$ ice surface elevation
  \item $b=$ bedrock surface elevation
  \item $T=$ temperature
  \item $\mathbf{u}=(u,v,w)=$ ice velocity
  \item $\rho=$ density of ice
  \item $\rho_w=$ density of ocean water
  \item $g=$ acceleration of gravity
  \item $n$ Glen flow law exponent (usually $=3$)
  \item $A=A(T)=$ ice softness in Glen law ($\mathbf{D}_{ij} = A(T) \tau^{n-1} \tau_{ij}$)
  \item \alert{please ask about notation!}
  \end{itemize}

\end{frame}


\begin{frame}{Matlab/Octave codes}

\begin{itemize}
\item lectures and notes are based on 18 Matlab/Octave codes
\item \dots a few will appear in these lectures
\item each is $\sim$ 1/2 page
\item please give them a try! see \texttt{mfiles/} directory at

\bigskip
  \centerline{\fbox{\url{https://github.com/bueler/mccarthy}}}
\end{itemize}
\end{frame}


\section[introduction]{introduction: a view from outside glaciology}

\begin{frame}{ice in glaciers is a \emph{fluid}}

\begin{itemize}
\item what's a fluid?
\item is it just a collection of particles?

\medskip
\item<2-3> its a mathematical abstraction

\medskip
\item<3> \dots with equations relating these fields:
  \begin{itemize}
  \item[$\circ$] a scalar \emph{density}\quad $\rho(t,x,y,z)$
  \item[$\circ$] a scalar \emph{pressure}\quad $p(t,x,y,z)$
  \item[$\circ$] a vector \emph{velocity}\quad $\mathbf{u}(t,x,y,z)$
  \end{itemize}
\end{itemize}

\vspace{-2mm}
\includegraphics[width=0.30\textwidth]{liquid} \hfill \uncover<2-3>{\includegraphics[width=0.23\textwidth]{lighterfluidalpha}} \hfill
\uncover<3>{\includegraphics[width=0.4\textwidth]{polaris}}
\end{frame}


\begin{frame}{ice in glaciers is an atypical fluid}

\begin{itemize}
\item if the ice were
  \begin{itemize}
  \item[$\circ$] faster-moving than it actually is, and
  \item[$\circ$] linearly-viscous like liquid water
  \end{itemize}
  
  then it would be a ``typical'' fluid

\bigskip
\item for typical fluids one uses the Navier-Stokes equations:
\begin{align*}
\nabla \cdot \mathbf{u} &= 0 &&\text{\emph{incompressibility}} \\
\rho \left(\mathbf{u}_t + \mathbf{u}\cdot\nabla \mathbf{u}\right) &= -\nabla p + \nabla \cdot \tau_{ij} + \rho \mathbf{g} &&\text{\emph{stress balance}} \\
2 \nu D\mathbf{u}_{ij} &= \tau_{ij} &&\text{\emph{flow law}}
\end{align*}

\medskip
    \begin{itemize}
    \item[$\circ$] stress balance equation is ``$m a = F$''
    \item[$\circ$] already a subtle model
    \end{itemize}
\end{itemize}
\end{frame}


\begin{frame}{glaciology as computational fluid dynamics}

\begin{itemize}
\item \alert{yes}, numerical ice sheet flow modelling is ``computational fluid dynamics''
  \begin{itemize}
  \item[$\circ$] it's large-scale like atmosphere and ocean
  \item[$\circ$] \dots\, but it is a weird one
  \end{itemize}
\item consider what makes atmosphere/ocean flow exciting:
  \begin{itemize}
  \item[$\circ$] turbulence
  \item[$\circ$] convection
  \item[$\circ$] coriolis force
  \item[$\circ$] density stratification
  \end{itemize}
\item none of the above list is relevant to ice flow
\item what could be interesting about the flow of slow, cold, stiff, laminar, inert old ice?
  \begin{itemize}
  \item[$\circ$] \emph{ice dynamics!}
  \end{itemize}
\end{itemize}
\end{frame}


\begin{frame}{ice is a slow, shear-thinning fluid}

\begin{itemize}
\item ice fluid is \emph{slow} and \emph{non-Newtonian}
    \begin{itemize}
    \item[$\circ$] ``slow'' is a technical term:
      $$\rho \left(\mathbf{u}_t + \mathbf{u}\cdot\nabla \mathbf{u}\right) \approx 0 \qquad \iff \qquad \begin{pmatrix} \text{forces of inertia} \\ \text{are neglected} \end{pmatrix}$$
    \item[$\circ$] ice is non-Newtonian in a ``shear-thinning'' way
        \begin{itemize}
        \item higher strain rates means lower viscosity
        \item viscosity $\nu$ is not constant
        \end{itemize}
    \end{itemize}

\bigskip
\item thus the standard model is Glen-law Stokes:
\begin{align*}
\nabla \cdot \mathbf{u} &= 0 &&\text{\emph{incompressibility}} \\
0 &= - \nabla p + \nabla \cdot \tau_{ij} + \rho\, \mathbf{g} &&\text{\emph{stress balance}} \\
D\mathbf{u}_{ij} &= A \tau^{n-1} \tau_{ij} &&\text{\emph{flow law}}
\end{align*}

\end{itemize}
\end{frame}


\begin{frame}{``slow'' means no memory of velocity/momentum}

\begin{itemize}
\item note \emph{no time derivatives} in Stokes model:
\small
\begin{align*}
\nabla \cdot \mathbf{u} &= 0 \\
0 &= - \nabla p + \nabla \cdot \tau_{ij} + \rho\, \mathbf{g} \\
D\mathbf{u}_{ij} &= A \tau^{n-1} \tau_{ij}
\end{align*}
\normalsize
\item thus a time-stepping ice sheet code can/must recompute the full velocity field at every time step
  \begin{itemize}
  \item[$\circ$] which does not require velocity from the previous time step
  \end{itemize}
\item velocity is a ``diagnostic'' output not needed for starting or restarting the model
\end{itemize}
\end{frame}


\begin{frame}{plane flow Stokes}

\begin{itemize}
\item suppose we work in a $x,z$ plane
    \begin{itemize}
    \item[$\circ$] e.g.~glacier centerline or a cross-flow plane
    \end{itemize}
\item $n=3$ Stokes equations say:
\begin{empheq}[]{align}
u_x + w_z &= 0 &&\text{\emph{incompressibility}}\notag \\
p_x &= \tau_{11,x} + \tau_{13,z} &&\text{\emph{stress balance} ($x$)} \notag \\
p_z &= \tau_{13,x} - \tau_{11,z} - \rho g &&\text{\emph{stress balance} ($z$)} \notag \\
u_x &= A \tau^2 \tau_{11} &&\text{\emph{flow law (diagonal)}}\notag \\
u_z + w _x &= 2 A \tau^2 \tau_{13} &&\text{\emph{flow law (off-diagonal)}} \notag
\end{empheq}

\vspace{-2mm}
    \begin{itemize}
    \item[$\circ$] \emph{notation}: subscripts $x,z$ denote partial derivatives, $\tau_{13}$ is the ``vertical'' shear stress, $\tau_{11}$ and $\tau_{33}=-\tau_{11}$ are (deviatoric) longitudinal stresses
    \end{itemize}
\item we have five equations in five unknowns ($u,w,p,\tau_{11},\tau_{13}$)
\item this is complicated enough \dots what about in a simplified situation?
\end{itemize}
\end{frame}


\begin{frame}{slab-on-a-slope}

\hfill \includegraphics[width=0.4\textwidth]{slab}

\vspace{-30mm}
\begin{itemize}
\item suppose constant thickness
\item tilt bedrock by angle $\alpha$
\item rotate the coordinates
\item get replacement expressions:
\begin{align*}
\mathbf{g} &= g \sin\alpha\, \hat x - g \cos \alpha \,\hat z \phantom{dslfkj sdkfjlskdjf  sdlfj}\\
p_x &= \tau_{11,x} + \tau_{13,z} + \rho g \sin\alpha \\
p_z &= \tau_{13,x} - \tau_{11,z} - \rho g \cos\alpha
\end{align*}
\item for \alert{slab-on-a-slope} there is \emph{no variation in} $x$:\quad $\partial/\partial x = 0$
\item the equations simplify:
\small
\begin{empheq}[box=\fbox]{align}
w_z &= 0 &   0 &= \tau_{11} \notag \\
\tau_{13,z} &= - \rho g \sin\alpha &   u_z &= 2 A \tau^2 \tau_{13} \notag \\
p_z &= - \rho g \cos\alpha \notag
\end{empheq}
\end{itemize}

\end{frame}


\begin{frame}{slab-on-a-slope 2}

\begin{itemize}
\item add boundary conditions:
	$$w(\text{base})=0, \qquad p(\text{surface})=0, \qquad u(\text{base})=u_0$$
\item by integrating vertically, get:
\begin{align*}
w &= 0 \phantom{asdfklj asldkfjalk asdfkj sdlfkj sldafkj adlfjl sdfakj }\\
p &= \rho g \cos\alpha (H-z) \\
\tau_{13} &= \rho g \sin\alpha (H-z)
\end{align*}

\vspace{-25mm}
\hfill \includegraphics[width=0.4\textwidth]{slabshear}

\vspace{-7mm}
\item $\tau_{13}$ is linear in depth

\medskip
\item from $u_z = 2 A \tau^2 \tau_{13}$ get \alert{velocity formula}
\vspace{-0.05in}
\begin{align*}
u(z) &= u_0 + 2 A (\rho g \sin\alpha)^3 \int_0^z (H-z')^3\,dz' \\
     &= u_0 + \frac{1}{2} A (\rho g \sin\alpha)^3  \left(H^4 - (H-z)^4\right)
\end{align*}
\end{itemize}
\end{frame}


\begin{frame}{slab-on-a-slope 3}

\begin{columns}
\begin{column}{0.6\textwidth}
\begin{itemize}
\item do we believe these equations?
\item velocity formula on last slide gives figure below
\item compare to observations at right
\end{itemize}
\begin{center}
% NOT preserving aspect ratio
\includegraphics[width=0.6\textwidth,height=0.5\textheight]{slabvel}
\end{center}
\end{column}

\begin{column}{0.4\textwidth}
\includegraphics[width=1.0\textwidth]{athabasca-deform}

\medskip
\scriptsize
Velocity profile of the Athabasca Glacier, Canada, derived from inclinometry (Savage and Paterson, 1963)
\end{column}
\end{columns}
\end{frame}


\begin{frame}{mass conservation}

\small
\begin{itemize}
\item[Q:] having computed the velocity $\mathbf{u}$ \dots so what?
\item[A:] velocity determines changing ice shape through \alert{mass conservation}
\item suppose flowline ice has variable thickness $H(t,x)$
\item define the vertical average of velocity:
	$$\bar U(t,x) = \frac{1}{H}\int_0^{H} u(t,x,z)\,dz \phantom{sdlj asdlbj asldbfj asdlfj}$$

\vspace{-20mm}
\hfill \includegraphics[width=0.3\textwidth]{slabmasscontfig}
\item with climatic mass balance $M(x)$, consider change of area in the figure:
	$$\frac{dA}{dt} = \int_{x_1}^{x_2} M(x)\,dx + \bar U_1 H_1 - \bar U_2 H_2$$

    \vspace{-2mm}
    \begin{itemize}
    \item[$\circ$] area in 2D becomes volume in 3D
    \end{itemize}
\item assume width $dx=x_2-x_1$ is small, so $A\approx dx\, H$, and divide by $dx$ to get the  \emph{mass continuity (conservation) equation}
   $$H_t = M - \left(\bar U H\right)_x$$
\end{itemize}
\end{frame}


\begin{frame}{combine equations so far}

\begin{itemize}
\item from slab-on-slope velocity formula in $u_0=0$ case, get flux:
\begin{align*}
q = \bar U H &= \int_0^H \frac{1}{2} A (\rho g \sin\alpha)^3  \left(H^4 - (H-z)^4\right)\,dz \\
	&= \frac{2}{5} A (\rho g \sin\alpha)^3 H^5
\end{align*}
\item note $\sin \alpha \approx \tan\alpha = - h_x$
\item combine with mass continuity $H_t = M - \left(\bar U H\right)_x$ to get:
  $$H_t = M + \left(\frac{2}{5} (\rho g)^5 A H^5 |h_x|^2 h_x\right)_x$$

\medskip
\item this is the ``shallow ice approximation'' (SIA)
    \begin{itemize}
    \item[$\circ$] a very rough explanation of it!
    \end{itemize}
\end{itemize}
\end{frame}



\section{shallow ice sheets}

\begin{frame}{slow, non-Newtonian, shallow, and sliding}

\begin{itemize}
\item ice sheets have four outstanding properties \emph{as fluids}:
  \begin{enumerate}
  \item slow
  \item non-Newtonian
  \item shallow (often)
  \item contact slip (sometimes)
  \end{enumerate}
\end{itemize}
\end{frame}


\begin{frame}{regarding ``shallow''}

\begin{itemize}
\item in \alert{red} is a Greenland cross section
  \begin{itemize}
  \item[$\circ$] at $71^\circ$ N: \qquad width = 760 km, thickness = 3200 m
  \end{itemize}
\item green and blue: standard vertically-exaggerated cross section
\end{itemize}

\begin{center}
  \includegraphics[width=0.85\textwidth]{green-transect}
\end{center}
\end{frame}


\begin{frame}{flow model I: shallow ice approximation (SIA)}

a model which applies to
\begin{itemize}
\item small depth-to-width ratio (``shallow'') grounded ice sheets
\item for \emph{not}\, rough/steep bed topography
\item when flow is \emph{not}\, dominated by sliding at the base
\end{itemize}

\begin{center}
  \includegraphics[width=0.7\textwidth]{polaris}

\tiny ``Polaris Glacier,'' northwest Greenland, photo 122, Post \& LaChapelle (2000)
\end{center}

\end{frame}


\begin{frame}{SIA model equations}

\begin{itemize}
\item simple slogan: \emph{the SIA uses the formulas from slab-on-a-slope}
  \begin{itemize}
  \item[$\circ$] better explanation: expand the Stokes equations in aspect ratio
  \end{itemize}

\item shear stress approximation:
	$$(\tau_{13},\tau_{23}) = - \rho g (h-z) \nabla h$$
\item then horizontal velocity $\mathbf{u} = (u,v)$ satisfies this approximation:
\begin{align*}
\mathbf{u}_z &= 2 A |(\tau_{13},\tau_{23})|^{n-1} (\tau_{13},\tau_{23}) \\
     &= - 2 A (\rho g)^n (h-z)^n |\nabla h|^{n-1} \nabla h
\end{align*}
\item by integrating vertically (non-sliding case):
    $$\mathbf{u} = - \frac{2 A (\rho g)^n}{n+1} \left[H^{n+1} - (h-z)^{n+1}\right] |\nabla h|^{n-1} \nabla h$$
\item integrate again to get $\overline{\mathbf{u}}$
\item add mass continuity: \quad $H_t = M - \left(\overline{\mathbf{u}} H\right)_x$
\end{itemize}
\end{frame}


\begin{frame}{SIA thickness equation}

\begin{itemize}
\item thus the non-sliding, isothermal SIA for thickness evolution:
\begin{empheq}[box=\fbox]{equation}
H_t = M + \Div \left(\Gamma H^{n+2} |\grad h|^{n-1} \grad h \right) \label{sia}
\end{empheq}

\vspace{-2mm}
  \begin{itemize}
  \item[$\circ$] $H$ is ice thickness, $h$ is ice surface elevation
  \item[$\circ$] \dots also $b$ is bed elevation and $h=H+b$
  \item[$\circ$] $M$ is surface (and basal) mass balance
  \item[$\circ$] $\Gamma = 2 A (\rho g)^n / (n+2)$ is a constant
  \end{itemize}
\item \begin{minipage}[t]{0.53\textwidth}
numerically solve (1) and you have a usable model for boring ice sheets
  \begin{itemize}
  \item[$\circ$] Barnes ice cap modeled by Mahaffy (1976) $\to$
  \end{itemize}
\end{minipage}
\end{itemize}
\bigskip \bigskip

\noindent good questions:
\begin{itemize}
\item[] how to solve (1) numerically?
\item[] how to \emph{think} about it?
\end{itemize}

\vspace{-43mm}
\hfill \includegraphics[width=0.34\textwidth]{mahaffy-profiles}
\end{frame}


\begin{frame}{heat equation}
\label{slide:heatcompare}

\small
\begin{columns}
\begin{column}{0.55\textwidth}
\begin{itemize}
\item to understand the SIA, compare to the \emph{heat equation}
\item recall Newton's law of cooling
	$$\frac{dT}{dt} = -K (T-T_{\text{ambient}})$$

  \begin{itemize}
  \item[$\circ$] $T$ is coffee temperature
  \item[$\circ$] $K$ from heat conductivity/capacity
  \end{itemize}
\item Newton's law for segments of a rod has $T_{\text{ambient}}$ as neighbor average:
\begin{align*}
\frac{dT_j}{dt} &= -K \left(T_j - \frac{1}{2} (T_{j-1} + T_{j+1}) \right) \\
	&= \frac{K}{2} \left(T_{j-1} - 2 T_j + T_{j+1}\right) 
\end{align*}
\item shrink segments to get \emph{heat equation}:
	$$T_t = D T_{xx}$$
\end{itemize}
\end{column}

\begin{column}{0.45\textwidth}
\hfill \includegraphics[width=0.4\textwidth]{coffee}

\vspace{13mm}
\includegraphics[width=1.02\textwidth]{heatconduction}
\end{column}
\end{columns}
\end{frame}


\begin{frame}{major analogy: SIA versus 2D heat equation}

\begin{itemize}
\item general 2D heat eqn: \quad $T_t = F + \Div (D\, \grad T)$
  \begin{itemize}
  \item[$\circ$] $F(t,x,y)$ is heat source function and $D>0$ is diffusivity
  \end{itemize}
\item side-by-side comparison:

\medskip
\begin{tabular}{cc}
\scriptsize SIA for ice thickness \, $H(t,x,y)$ & \scriptsize heat eqn for temperature $T(t,x,y)$ \normalsize \medskip \\
	\hspace{-6mm} $H_t = M + \Div \left({\color{red}\Gamma H^{n+2} |\grad h|^{n-1}}\, \grad h \right)$  &  $T_t = F + \Div (D\, \grad T)$
\end{tabular} 

\medskip
\item identify the diffusivity in the SIA:
	$$D = {\color{red}\Gamma H^{n+2} |\grad h|^{n-1}}$$
\item non-sliding shallow ice flow \emph{diffuses} the ice sheet geometry
\item considerations when using this analogy:
  \begin{itemize}
  \item[$\circ$]  $D$ is not constant; it depends on the solution $H(t,x,y)$
  \item[$\circ$]  $D\to 0$ at margin, where $H\to 0$
  \item[$\circ$]  $D\to 0$ at divides/domes, where $|\grad h|\to 0$
  \end{itemize}
\end{itemize}
\end{frame}


\begin{frame}{numerics for heat equation: basic finite differences}

\begin{itemize}
\item numerical schemes for heat equation are good start for SIA
\item these slides only consider \emph{finite difference} (FD) schemes

\bigskip
\item first, for differentiable $f(x)$ \emph{Taylor's theorem} says
	$$f(x+h) = f(x) + f'(x) h + \frac{1}{2} f''(x) h^2 + \frac{1}{3!} f'''(x) h^3 + \dots$$
\normalsize
\item you can replace ``$h$'' by multiples of $\Delta x$, e.g.:
\small
\begin{align*}
f(x-\Delta x) &= f(x) - f'(x) \Delta x + \frac{1}{2} f''(x) \Delta x^2 - \frac{1}{3!} f'''(x) \Delta x^3 + \dots \\
f(x+2\Delta x) &= f(x) + 2 f'(x) \Delta x + 2 f''(x) \Delta x^2 + \frac{4}{3} f'''(x) \Delta x^3 + \dots
\end{align*}
\normalsize
\item \emph{main idea for FD schemes}:  combine expressions like these to approximate derivatives from the values of functions on a grid
\end{itemize}
\end{frame}


\begin{frame}{partial derivatives}

\begin{itemize}
\item we want FD approximations of partial derivatives
\item for example, with any function $u=u(t,x)$:
\small
\begin{align*}
u_t(t,x) &= \frac{u(t+\Delta t,x) - u(t,x)}{\Delta t} + O(\Delta t), \\
u_t(t,x) &= \frac{u(t+\Delta t,x) - u(t-\Delta t,x)}{2\Delta t} + O(\Delta t^2), \\
u_x(t,x) &= \frac{u(t,x+\Delta x) - u(t,x-\Delta x)}{2\Delta x} + O(\Delta x^2), \\
u_{xx}(t,x) &= \frac{u(t,x+\Delta x) - 2 u(t,x) + u(t,x-\Delta x)}{\Delta x^2} + O(\Delta x^2)
\end{align*}
\normalsize
\item sometimes we want a derivative in-between grid points:
\small
	$$u_x(t,x+\tfrac{\Delta x}{2}) = \frac{u(t,x+\Delta x) - u(t,x)}{\Delta x} + O(\Delta x^2)$$
\normalsize
\item \emph{note}: ``$+O(h^2)$'' is better than ``$+O(h)$'' if $h$ is a small number
\end{itemize}
\end{frame}


\begin{frame}{explicit scheme for heat equation}
\label{slide:explicit}

\begin{itemize}
\item consider 1D heat equation $T_t = D T_{xx}$
\item thus, approximately:
\small
	$$\frac{T(t+\Delta t,x) - T(t,x)}{\Delta t} \approx D\,\frac{T(t,x+\Delta x) - 2 T(t,x) + T(t,x-\Delta x)}{\Delta x^2}$$

\normalsize
    \begin{itemize}
    \item[$\circ$] make it equality for a \emph{scheme}
    \end{itemize} 
\item the difference between $T_t = D T_{xx}$ and the scheme is $O(\Delta t,\Delta x^2)$
\item notation:
    \begin{itemize}
    \item[$\circ$] $(t_n,x_j)$ is a point in the time-space grid
    \item[$\circ$] $T_j^n \approx T(t_n,x_j)$  \only<2>{\hfill \alert{$\gets \quad T_j^n$ from the scheme, $T(t_n,x_j)$ generally unknown}}
    \end{itemize} 
\item let $\mu = D \Delta t / (\Delta x)^2$, so this scheme is
\small
	$$T_j^{n+1} = \mu T_{j+1}^n + (1 - 2 \mu) T_j^n + \mu T_{j-1}^n \phantom{sdfkj sdkfj asdlfj asldkfj asdflkj}$$
\normalsize
\item \emph{stencil} of this \emph{explicit} scheme at right \large $\to$ \normalsize
\end{itemize}

\vspace{-15mm}
\hfill \includegraphics[width=0.28\textwidth]{expstencil}
\end{frame}


\begin{frame}{explicit scheme in 2D}

\begin{itemize}
\item recall 2D heat equation:
    $$T_t = D(T_{xx} + T_{yy})$$
\item again: $T_{jk}^n \approx T(t_n,x_j,y_k)$
\item the explicit scheme is
\small
	$$\frac{T_{jk}^{n+1} - T_{jk}^n}{\Delta t} = D\,\left(\frac{T_{j+1,k}^n - 2 T_{jk}^n + T_{j-1,k}^n}{\Delta x^2} + \frac{T_{j,k+1}^n - 2 T_{jk}^n + T_{j,k-1}^n}{\Delta y^2}\right)$$
\normalsize
\item which becomes a computable iteration:
\small
\begin{align*}
T_{jk}^{n+1} &= T_{jk}^n + \mu_x (T_{j+1,k}^n - 2 T_{jk}^n + T_{j-1,k}^n) \hspace{40mm} \\
             &\quad + \mu_y (T_{j,k+1}^n - 2 T_{jk}^n + T_{j,k-1}^n)
\end{align*}
\normalsize
where $\mu_x=D\Delta t/\Delta x^2$ and $\mu_y = D\Delta t/\Delta y^2$
\end{itemize}

\vspace{-20mm}
\hfill \includegraphics[width=0.3\textwidth]{exp2dstencil}
\end{frame}


\begin{frame}{implementation}
\label{slide:heatmatlab}

\minput{heat}

\small
\begin{itemize}
\item solves $T_t = D(T_{xx} + T_{yy})$ on the square $-1 < x < 1$, $-1 < y < 1$
\item code uses ``colon notation'' to remove loops (over space)
\item example (next slide) uses initial condition $T_0(x,y) = e^{-30 r^2}$
\end{itemize}
\end{frame}


\begin{frame}{the look of success}

\begin{itemize}
\item[] \texttt{>>  heat(1.0,30,30,0.001,20)}

\medskip
solves $T_t = D(T_{xx} + T_{yy})$ for $T(t,x,y)$ with $D=1$, $30\times 30$ grid, $\Delta t = 0.001$, and $N=20$ time steps
\end{itemize}

\bigskip
\begin{columns}
\begin{column}{0.5\textwidth}
initial condition $T(0,x,y)$

\bigskip
\begin{center}
\includegraphics[width=1.0\textwidth]{initialheat}
\end{center}
\end{column}
\begin{column}{0.5\textwidth}
\mbox{approximate solution $T(0.02,x,y)$}

\bigskip
\begin{center}
\includegraphics[width=1.0\textwidth]{finalheat}
\end{center}
\end{column}
\end{columns}
\end{frame}


\begin{frame}{the look of instability}

results from solving $T_t = D(T_{xx} + T_{yy})$ on the \emph{same} space grid and at the \emph{same} time, but with slightly-different time steps:

\bigskip
\begin{columns}
\begin{column}{0.5\textwidth}
\begin{center}
\includegraphics[width=1.0\textwidth]{stability}

\uncover<2->{$$\frac{D\Delta t}{\Delta x^2}= 0.2$$}
\end{center}
\end{column}
\begin{column}{0.5\textwidth}
\begin{center}
\includegraphics[width=1.0\textwidth]{instability}

\uncover<2->{$$\frac{D\Delta t}{\Delta x^2}= 0.4$$}
\end{center}
\end{column}
\end{columns}
\end{frame}


\begin{frame}{avoid the instability}
\label{slide:maxprinc}

\begin{itemize}
\item recall 1D explicit scheme had the form 
	$$T_j^{n+1} = \mu T_{j+1}^n + (1 - 2 \mu) T_j^n + \mu T_{j-1}^n$$
\item thus the new value $u_j^{n+1}$ is an \emph{average} of the old values \emph{if the middle coefficient is positive}:
	$$1 - 2 \mu \ge 0 \quad \iff \quad  \frac{D\Delta t}{\Delta x^2} \le \frac{1}{2} \quad \iff \quad \Delta t \le \frac{\Delta x^2}{2 D}$$
    \begin{itemize}
    \item[$\circ$] averaging is stable because averaged wiggles are smaller than wiggles
    \item[$\circ$] this condition is a sufficient \emph{stability criterion}
    \end{itemize}
\item instabilities arise because \emph{the time step is too big}
\end{itemize}
\end{frame}


\begin{frame}{\textsl{adaptive} implementation: guaranteed stability}

\minput{heatadapt}

\begin{itemize}
\item same as \texttt{heat.m} except:

\begin{center}
\emph{choose time step from stability criterion}
\end{center}
\end{itemize}\end{frame}


\begin{frame}{alternative instability fix: implicitness}

\begin{itemize}
\item \alert{implicit} methods can be stable for \emph{any} positive time step $\Delta t$
\item example is \emph{Crank-Nicolson} scheme $\longrightarrow$

\vspace{-7mm}
\hfill \includegraphics[width=0.35\textwidth]{cnstencil}

\vspace{-5mm}
    \begin{itemize}
    \item[$\circ$] has smaller error too: $O(\Delta t^2,\Delta x^2)$
    \end{itemize}
\item \emph{but} you must solve linear (or nonlinear) systems of equations to take each time step

\bigskip

\small
\item Donald Knuth has advice for ice sheet modelers:

\begin{center}
\emph{forget about small efficiencies \dots premature optimization is the root of all evil}
\end{center}

    \begin{itemize}
    \item[$\circ$] these slides use an explicit adaptive scheme for the SIA
    \item[$\circ$] most ice sheet models are explicit
    \item[$\circ$] Bueler (2016) is an exception: fully-implicit 
    \end{itemize}
\end{itemize}
\end{frame}


\begin{frame}{variable diffusivity and staggered grids}

\begin{itemize}
  \item SIA has diffusivity which varies in space
  \item consider the generalized diffusion/heat equation for analogy:
     $$T_t = F + \Div \big(D(x,y) \grad (T+b)\big)$$
  \item an explicit method is conditionally stable if we evaluate diffusivity $D(x,y)$ at \alert{staggered} grid points:
  \small
\begin{align*}
\Div \left(D(x,y) \grad X\right) &\approx \frac{D_{j+1/2,k}(X_{j+1,k} - X_{j,k}) - D_{j-1/2,k}(X_{j,k} - X_{j-1,k})}{\Delta x^2} \\
	&\qquad + \frac{D_{j,k+1/2}(X_{j,k+1} - X_{j,k}) - D_{j,k-1/2}(X_{j,k} - X_{j,k-1})}{\Delta y^2}
\end{align*}
\normalsize
where $X=T+b$
\item in stencil at right $\longrightarrow$
    \begin{itemize}
    \item[] diamonds: $T,b$
    \item[] triangles: $D$
    \end{itemize}

\vspace{-15mm}
\hfill \includegraphics[width=0.3\textwidth]{diffstencil}
\end{itemize}
\end{frame}


\begin{frame}
  \frametitle{general diffusion equation code}

\minputtiny{diffstag}

\small
\begin{itemize}
\item solves abstract diffusion equation $T_t = \Div \left(D(x,y)\, \grad (T+b)\right)$
\item user supplies diffusivity on staggered grid
\end{itemize}
\end{frame}


\begin{frame}{computing diffusivity in SIA}

\begin{itemize}
\item we need SIA diffusivity $D = \Gamma H^{n+2} |\grad h|^{n-1}$ on the staggered grid
\item various schemes proposed: Mahaffy (1976), Hindmarsh and Payne (1996), Bueler (2016)
\item all schemes seek accurate diffusivity by:
  \begin{itemize}
  \item[$\circ$] averaging thickness $H$
  \item[$\circ$] differencing surface elevation $h$
  \item[$\circ$] in a ``balanced'' way
  \end{itemize}
\item Mahaffy stencil is compact $\longrightarrow$
\end{itemize}

\vspace{-10mm}
\hfill  \includegraphics[width=0.3\textwidth]{mahaffystencil}
\end{frame}


\begin{frame}
  \frametitle{SIA implementation: flat bed case}

\minputtiny{siaflat}

\small
\begin{itemize}
\item calls \texttt{diffstag.m}
\end{itemize}
\end{frame}


\begin{frame}{interruption: verification}
\begin{itemize}
\item how do you make sure your \emph{implemented} numerical ice flow code is correct?
  \begin{itemize}
  \item[$\circ$] \emph{technique} 1 = \alert{infallibility}: don't make any mistakes
  \item[$\circ$] \emph{technique} 2 = \alert{intercomparison}: compare your model with others, and hope that the outliers are the ones with errors
  \item[$\circ$] \emph{technique} 3 = \alert{verification}: build-in a comparison to an exact solution, and actually measure the numerical error
  \end{itemize}

\medskip
\item where to get exact solutions for ice flow models?
  \begin{itemize}
  \item[$\circ$] textbooks: Greve and Blatter (2009), van der Veen (2013)
  \item[$\circ$] manufactured solns to thermo-coupled SIA (Bueler et al 2007)
  \item[$\circ$] flowline and cross-flow SSA solns (Bodvardsson, 1955; van der Veen, 1985; Schoof, 2006; Bueler 2014)
  \item[$\circ$] flowline Blatter solns (Glowinski and Rappaz 2003)
  \item[$\circ$] constant viscosity flowline Stokes solns (Ladyzhenskaya 1963, Balise and Raymond 1985)
  \item[$\circ$] manufactured solns to Stokes equations (Sargent and Fastook 2010; Jouvet and Rappaz 2011; Leng et al 2013)
  \end{itemize}
\end{itemize}
\end{frame}


\begin{frame}{the Green's function of the heat equation}

\begin{itemize}
\item recall heat equation in 1D with constant diffusivity $D>0$:
	$$T_t = D T_{xx}$$
\item many exact solutions are known!
\item \emph{time-dependent Green's function} solution shown below
  \begin{itemize}
  \item[$\circ$] a.k.a.~``fundamental solution'' or ``heat kernel''
  \item[$\circ$] starts at time $t=0$ with a ``delta function'' $T(0,x)=\delta_0(x)$
  \item[$\circ$] then spreads out over time
  \end{itemize}
\end{itemize}

\begin{center}
\includegraphics[width=0.5\textwidth]{heatscaling}

\emph{increasing time} \Large $\to$
\end{center}
\end{frame}


\begin{frame}{the Green's function of the heat equation, cont.}

\begin{itemize}
\item the Green's function can be found by a method which generalizes to the SIA: the solution is ``self-similar'' over time
\item as time goes it changes shape by
  \begin{itemize}
  \item[$\circ$] shrinking the output (vertical) axis and
  \item[$\circ$] lengthening the input (horizontal) axis
  \end{itemize}
\item \dots but otherwise it is the same shape
\item the integral over $x$ is independent of time
\end{itemize}

\begin{center}
\includegraphics[width=0.45\textwidth]{heatscaling}
\end{center}
\end{frame}


\begin{frame}{similarity solutions}

\begin{itemize}
\item ``similarity'' variables for the 1D heat equation are
	$$s \stackrel{\text{\emph{input scaling}}}{\phantom{\Big|}=\phantom{\Big|}} t^{-1/2} x, \qquad T(t,x) \stackrel{\text{\emph{output scaling}}}{\phantom{\Big|}=\phantom{\Big|}} t^{-1/2} \phi(s)$$
\item derive from ODE: \, $\phi(s) = C\, e^{-s^2/(4D)}$
\item thus the Green's function of heat equation in 1D is
	$$T(t,x) = C\, t^{-1/2} e^{-x^2/(4Dt)}$$

\bigskip
\item \begin{minipage}[t]{0.55\textwidth}
\noindent \emph{a tangent?}: Einstein (1905) discovered that the average distance traveled by particles in thermal motion scales like $\sqrt{t}$, so:

\centerline{$s = t^{-1/2}x$ is an invariant}
\end{minipage}
\end{itemize}

\vspace{-20mm}
\hfill \includegraphics[width=0.3\textwidth]{brownian}
\end{frame}


\begin{frame}{similarity solution to SIA}

\begin{itemize}
\item \emph{1981}:  Peter Halfar discovers the similarity solution of the SIA in the case of flat bed and zero surface mass balance
  \begin{itemize}
  \item[$\circ$] for 20 years, almost no one cares
  \end{itemize}
\item Halfar's 2D solution for $n=3$ Glen law has scalings
   $$H(t,r)=t^{-1/9} \phi(s), \qquad s = t^{-1/18} r$$
and a simple invariant shape function $\phi(s)$

\medskip
\item immediate conclusion (\alert{movie follows}): the diffusion of ice really slows down as the shape flattens out!
\end{itemize}
\end{frame}


\begin{frame}{Halfar solution: the movie}
\label{slide:plothalfar}

\medskip
\small
frames from $t=0.3$ to $t = 10^6$ years

\medskip
nearly equally-spaced in \emph{logarithmic} time

\begin{center}
\multiinclude[format=png,graphics={width=10cm}]{anim/halfar}
\end{center}
\end{frame}


\begin{frame}{Halfar solution: the formula}

\begin{itemize}
\item for $n=3$ the solution formula is:
  $$H(t,r) = H_0 \left(\frac{t_0}{t}\right)^{1/9} \left[1 - \left(\left(\frac{t_0}{t}\right)^{1/18} \frac{r}{R_0}\right)^{4/3}\right]^{3/7}$$
if $H_0$, $R_0$ are central height and ice cap radius
\item the ``characteristic time''
  $$t_0 = \frac{1}{18 \Gamma} \left(\frac{7}{4}\right)^3 \frac{R_0^4}{H_0^{7}}$$
\item it is a simple formula to use for verification!
    \begin{itemize}
    \item[$\circ$] it finally appears in a textbook: van der Veen (2013)
    \end{itemize}
\end{itemize}
\end{frame}


\begin{frame}{is the Halfar solution useful for modelling?}

\begin{itemize}
\item John Nye and others (2000) compared different flow laws for the South Polar Cap on Mars
\item they evaluated $\text{CO}_2$ ice and $\text{H}_2\text{O}$ ice softness parameters by comparing the long-time behavior of the corresponding Halfar solutions
\item conclusions:
  \begin{quote}
  \dots none of the three possible [$\text{CO}_2$] flow laws will allow a 3000-m cap, the thickness suggested by stereogrammetry, to survive for $10^7$ years, indicating that the south polar ice cap is probably not composed of pure $\text{CO}_2$ ice \dots the south polar cap probably consists of water ice, with an unknown admixture of dust
  \end{quote}
\item direct observation by landers and orbiters: yes indeed!
\end{itemize}

\end{frame}


\begin{frame}[fragile]
\frametitle{verifying SIA code vs Halfar}
\label{slide:verifysia}

\begin{columns}
\begin{column}{0.55\textwidth}
\begin{itemize}
\item run \texttt{siaflat.m} using Halfar solution for verification:
\scriptsize
\begin{verbatim}
>> verifysia(20)
average thickness error  = 22.310
maximum thickness error  = 227.845
>> verifysia(40)
average thickness error  = 9.459
maximum thickness error  = 240.941
>> verifysia(80)
average thickness error  = 2.771
maximum thickness error  = 153.845
>> verifysia(160)
average thickness error  = 1.085
maximum thickness error  = 104.605
\end{verbatim}
\normalsize

\medskip
\item how much does effort increase when halving the grid spacing?
\end{itemize}
\end{column}
\begin{column}{0.45\textwidth}
\includegraphics[width=1.0\textwidth]{siaerror}

\bigskip\medskip

\hfill \includegraphics[width=0.85\textwidth]{eismintone}

\scriptsize \hfill \emph{figure 2 in Huybrechts et al.~(1996)}
\end{column}
\end{columns}
\end{frame}


\begin{frame}{demonstrate robustness}

\begin{itemize}
\item \texttt{roughice.m} sets-up the nasty initial state at left
\item calls \texttt{siaflat.m} to evolve it for 50 years \dots adaptive time steps!
\item get familiar-looking dome at right
\end{itemize}

\begin{columns}
\begin{column}{0.5\textwidth}
\includegraphics[width=0.8\textwidth]{roughinitial}
\end{column}
\begin{column}{0.5\textwidth}
\hfill \includegraphics[width=0.8\textwidth]{roughfinal}
\end{column}
\end{columns}

\vspace{-2mm}
\begin{center}
\includegraphics[width=0.35\textwidth]{roughtimesteps}
\end{center}
\end{frame}


\begin{frame}{model the Antarctic ice sheet}

\normalsize
\begin{itemize}
\item modify \texttt{siaflat.m} into \texttt{siageneral.m}:
  \begin{itemize}
  \item[$\circ$] observed accumulation as surface mass balance,
  \item[$\circ$] allow non-flat bed (so $H\ne h$),
  \item[$\circ$] calve any ice that is floating
  \end{itemize}
\item only adds 10 new lines of code!  \quad  (makes a good exercise)
\item results from this \emph{toy} Antarctic flow model, a 2000 model year run on a $\Delta x=50$ km grid; runtime a few seconds
\end{itemize}

\bigskip

\begin{columns}
\begin{column}{0.4\textwidth}
\includegraphics[height=1.75in]{antinitial}
\end{column}
\begin{column}{0.55\textwidth}
\includegraphics[height=1.75in]{antfinal}
\end{column}
\end{columns}
\end{frame}



\section{mass continuity}

\begin{frame}{the most-basic shallow assumption}

\begin{columns}

\begin{column}{0.6\textwidth}
\begin{itemize}
\item there are many shallow theories: SIA, SSA, hybrids, Blatter, \dots
\item \emph{all} make one assumption not required in Stokes:

\begin{center}
\alert{the surface and base of the ice are given by functions}
\end{center}
    \begin{itemize}
    \item[$\circ$] namely $z=h(t,x,y)$ and $z=b(t,x,y)$
    \item[$\circ$] surface overhang is not allowed
    \item[$\circ$] most numerical Stokes models in glaciology also make this assumption
    \end{itemize}
\end{itemize}
\end{column}

\begin{column}{0.4\textwidth}
\includegraphics[width=1.0\textwidth]{sshape}

\scriptsize
\begin{center}
\emph{not shallow!}
\end{center}
\vspace{6mm}

\includegraphics[width=1.0\textwidth]{serac}

\begin{center}
\emph{not shallow! not even a fluid!}
\end{center}
\end{column}
\end{columns}
\end{frame}


\begin{frame}{three equations for geometry change}

\begin{itemize}
\item let $a$ be the surface (climatic) mass balance function
    \begin{itemize}
    \item[$\circ$] $a>0$ is accumulation and $a<0$ is ablation
    \end{itemize}
\item let $s$ be the basal mass balance function (basal melt rate)
    \begin{itemize}
    \item[$\circ$] $s>0$ is basal melting and $s<0$ is freeze-on
    \end{itemize}
\item let $M=a-s$
    \begin{itemize}
    \item[$\circ$] ``climatic-basal mass balance function'' in glossary
    \end{itemize}
\item define the map-plane flux of ice,
	$$\bq = \int_{b}^{h} (u,v)\,dz = \overline{\mathbf{u}}\,H$$
\item the three equations for glacier geometry change:
\begin{align*}
&\text{surface kinematical} && h_t = a - u\big|_h h_x - v\big|_h h_y + w\big|_h  \\
&\text{base kinematical} && b_t = s - u\big|_b b_x - v\big|_b b_y + w\big|_b  \\
&\text{mass continuity} && H_t = M - \Div \bq
\end{align*}
\end{itemize}
\end{frame}


\begin{frame}{kinematic and mass continuity equations}

\begin{itemize}
\item[Q:] what does the ``most-basic shallow assumption'' get you?
\item[A:] for incompressible ice, any two equations imply the third:
\small
\begin{align*}
&\text{surface kinematical} && h_t = a - u\big|_h h_x - v\big|_h h_y + w\big|_h  \\
&\text{base kinematical} && b_t = s - u\big|_b b_x - v\big|_b b_y + w\big|_b  \\
&\text{mass continuity} && H_t = M - \Div \bq
\end{align*}
\normalsize

\bigskip
\item to show the equivalences:
  \begin{itemize}
  \item[$\circ$]  recall the incompressibility of ice
    $$u_x + v_y + w_z = 0$$
  \item[$\circ$]  use the Leibniz rule for differentiating integrals
  {\scriptsize
    $$\frac{d}{dx}\left(\int_{g(x)}^{f(x)} h(x,y)\,dy\right) = f'(x) h(x,f(x)) - g'(x) h(x,g(x)) + \int_{g(x)}^{f(x)} h_x(x,y)\,dy$$}
  \item[$\circ$]  it's an exercise
  \end{itemize}
\end{itemize}
\end{frame}


\begin{frame}{the mass continuity equation: a summary}

\begin{itemize}
\item most ice sheet models use the mass continuity equation and the base kinematical equation

\bigskip
\item regarding how to think about the \emph{mass continuity equation}
  $$H_t = M - \nabla \cdot (\overline{\mathbf{u}} H),$$
a summary:
  \begin{itemize}
  \item[$\circ$] its character depends on the stress balance
  \item[$\circ$] it \emph{is} a transport equation
  \item[$\circ$] \dots but it is a diffusion for frozen bed, large scale flows (SIA)
      \begin{itemize}
      \item if your fancy Stokes model is not diffusive in this case \dots \emph{it's wrong}
      \end{itemize}
  \item[$\circ$] it is not very diffusive for membrane stresses and low basal resistance (e.g.~SSA for ice shelves)
  \item[$\circ$] additional mass continuity equations are needed for liquid water on surface and base \dots with much shorter time scales \dots which gets complicated
  \end{itemize}

\medskip
\item there is \emph{not} much helpful theory on the transport problems in glaciology \dots maybe you will help find this theory!
\end{itemize}
\end{frame}


\begin{frame}{standard recipe for ice sheet models}

\begin{enumerate}
  \item use stress balance to compute velocity
      \begin{itemize}
      \item[$\circ$] often: get $(u,v)$, then compute $w$ from incompressibility
      \end{itemize}
  \item do other processes: thermodynamics, basal melt, calving, \dots
  \item decide on time-step $\Delta t$ from diffusivity $D$ \hfill (or: \emph{fixed} $\Delta t$, sadly)
  \item from velocity, surface balance, and base balance do time-step of mass continuity equation to get $H_t$
  \item update surface elevation: $h \gets h+H_t \Delta t$
  \item repeat at 1.
\end{enumerate}

\bigskip
\begin{itemize}
\item this paradigm is \alert{explicit time stepping of the whole model}
  \begin{itemize}
  \item[$\circ$]  it will always be low-performance
  \end{itemize}
\end{itemize}
\end{frame}



\section{shelves and streams}

\begin{frame}{flow model II: shallow shelf approximation (SSA)}
  
SSA model applies very well to \alert{ice shelves}
\begin{itemize}
\item \dots for parts away from grounding lines
\item \dots and away from calving fronts
\end{itemize}

\bigskip

\begin{center}
  \includegraphics[width=0.7\textwidth]{iceshelfedge}

\tiny Ekstr\"om ice shelf (Hans Grobe)
\end{center}
\end{frame}


\begin{frame}{application of SSA to ice streams}

SSA also applies reasonably well to \alert{ice streams}
\begin{itemize}
\item \dots with modest bed topography
\item \dots and weak bed strength\footnote{energy conservation (esp.~ice temperature and basal melt rate) and subglacial hydrology (esp.~subglacial water pressure) are major aspects of ice stream flow \dots but not addressed here!}
\item limited applicability near shear margins and grounding lines
\end{itemize}

\begin{center}
  \includegraphics[width=0.5\textwidth]{siple}

\tiny surface velocity for Siple Coast ice streams, Antarctica 
\end{center}
\end{frame}


\begin{frame}{but what is, \emph{and is not}, an ice stream?}

\begin{columns}
\begin{column}{0.4\textwidth}
\includegraphics[width=1.0\textwidth]{streamisbrae}

\bigskip
\scriptsize 
Jakobshavns Isbrae (\textbf{a}) and Whillans Ice Stream (\textbf{b}); plotted without vertical exaggeration (\tiny Truffer and Echelmeyer 2003)\scriptsize)
\end{column}
\begin{column}{0.6\textwidth}
\begin{itemize}
\item[a.] outlet glaciers:
  \begin{itemize}
  \item fast surface speed (up to $10 \,\text{km}\,\text{a}^{-1}$)
  \item uncertain how much is sliding
  \item vertical shear in much of ice column
  \item \emph{not} flat bed topography
  \item soft temperate ice may play big role
  \item \alert{shallow simplifications are dubious}
  \end{itemize}

\medskip
\item[b.] ice streams:
  \small
  \begin{itemize}
  \item fast surface speed (up to $1 \,\text{km}\,\text{a}^{-1}$)
  \item very concentrated vertical shear in thin layer near base (``sliding'')
  \item cold ice nearly down to bed
  \item SSA is accurate
  \end{itemize}
  \normalsize
\end{itemize}
\end{column}
\end{columns}
\end{frame}


\begin{frame}{SSA stress balance equation}

\begin{itemize}
\item only plane flow case (``flow line'') shown here
\item the stress balance equation, ice stream case:
\begin{empheq}[box=\fbox]{equation}
  \left({\color{red}2 A^{-1/n} H |u_x|^{1/n - 1} u_x}\right)_x - {\color{blue}C|u|^{m-1}u} = {\color{green}\rho g H h_x} \label{ssa}
\end{empheq}
\item the {\color{red} red term} inside parentheses is the vertically-integrated ``longitudinal'' or ``membrane'' stress
\item the {\color{blue} blue term} is basal resistance; $C=0$ in ice shelf
\item the {\color{green} green term} is  driving stress

\medskip
\item this equation determines velocity $u$
\item derived originally by Morland (1987), MacAyeal (1989)
\end{itemize}

\bigskip
\noindent once again, good questions:
\begin{enumerate}
\item[] how to solve (2) numerically?
\item[] how to \emph{think} about it?
\end{enumerate}
\end{frame}


\begin{frame}{flotation criterion and grounding line}

\begin{itemize}
\item the inequality ``$\rho H < - \rho_w b$'' is the \alert{flotation criterion}
  \begin{itemize}
  \item[$\circ$] assumes $z=0$ is sea level
  \item[$\circ$] grounding line $x=x_g$ is where $\rho H = - \rho_w b$
  \item[$\circ$] $H,u,u_x$ are all continuous at $x=x_g$
  \end{itemize}
\item surface elevation and driving stress according to flotation:
  \begin{itemize}
  \item[grounded:]
\begin{align*}
\rho H &> - \rho_w b \\
h &= H + b \\
\rho g H h_x &= \rho g H (H_x + b_x)
\end{align*}
  \item[floating:]
\begin{align*}
\rho H &< - \rho_w b \\
h &= (1-\rho/\rho_w) H \\
\rho g H h_x &= \rho(1-\rho/\rho_w) g H H_x
\end{align*}  
  \end{itemize}
\end{itemize}
\end{frame}


\begin{frame}{flow line model: from stream to shelf}
\label{slide:streamtoshelf}

\small
\begin{align*}
  u = u_0 & \qquad \text{ at } x = 0 \\
  \left.\begin{array}{r}
  \left(2 A^{-1/n} H |u_x|^{1/n - 1} u_x\right)_x - C|u|^{m-1}u = \rho g H h_x \\
  h = H + b
  \end{array}\right\}& \qquad \text{ on } 0 < x < x_g \\
  \left.\begin{array}{r}
  \left(2 A^{-1/n} H |u_x|^{1/n - 1} u_x\right)_x = \rho g H h_x \\
  h = (1-\rho/\rho_w) H
  \end{array}\right\}& \qquad \text{ on } x_g < x < x_c \\
  2 A^{-1/n} H |u_x|^{1/n - 1} u_x = \frac{1}{2}\rho (1-\rho/\rho_w) g H^2 & \qquad \text{ at } x = x_c
\end{align*}

\bigskip\bigskip
\begin{center}
  \includegraphics[width=0.7\textwidth]{flowline}
\end{center}
\end{frame}


\begin{frame}{steady ice shelf}

\begin{itemize}
\item I have limited goal here:

\begin{center}
\alert{describe a steady state, 1D ice shelf}
\end{center}
\item \emph{by-hand} result (van der Veen 1983): steady SSA ice shelf found from thickness $H_g$ and velocity $u_g$ at the grounding line

\begin{center}
  \includegraphics[width=0.35\textwidth]{steadyshelfprofile} \qquad \qquad
  \includegraphics[width=0.35\textwidth]{steadyshelfvelocity}
\end{center}

{\scriptsize \hspace{16mm} surface/base elevation \hspace{40mm} velocity}

\item we will use this to
  \begin{itemize}
  \item[$\circ$] understand the SSA better
  \item[$\circ$] verify a numerical SSA code
  \end{itemize}

\vspace{-9mm}
\item an ice shelf is a kind of fluid \emph{jet} \hfill \includegraphics[width=0.3\textwidth]{rocket-nozzle-expansion}
\end{itemize}
\end{frame}


\begin{frame}{numerical SSA stress balance example}

\begin{itemize}
\item fix the ice thickness $H(x)$ from the exact van der Veen formula
\item find the velocity numerically
\item verify using the exact van der Veen formula for velocity
\end{itemize}
\end{frame}


\begin{frame}{numerics of SSA stress balance}

\begin{itemize}
\item the stress balance is a nonlinear equation in the velocity:
  $$\left(2 A^{-1/n} H |u_x|^{1/n - 1} u_x\right)_x - C|u|^{m-1}u = \rho g H h_x$$

\vspace{-2mm}
    \begin{itemize}
    \item[$\circ$] thus \alert{iteration is needed}    
    \end{itemize}
\item coefficient ${\color{red} \bar \nu} = A^{-1/n} |u_x|^{1/n-1}$ is the ``effective viscosity'':
   $$\left(2 \,{\color{red} \bar \nu}\, H u_x\right)_x - C |u|^{m-1} u = \rho g H h_x$$
\item \emph{simplest iteration idea}: use old $\bar\nu$ to get new velocity solution, and repeat until things stop changing
  \begin{itemize}
  \item[$\circ$] this is ``Picard'' iteration  (versus Newton iteration, generally superior)
  \item[$\circ$] from last iterate $u^{(k-1)}$ define
     $$W^{(k-1)} = 2 \bar \nu H = 2 A^{-1/n} |u^{(k-1)}_x|^{1/n-1} H$$
  \item[$\circ$] solve for current iterate $u^{(k)}$:
     $$\left(W^{(k-1)} u^{(k)}_x\right)_x - C |u^{(k-1)}|^{m-1} u^{(k)} = \rho g H h_x$$
  \end{itemize}
\end{itemize}
\end{frame}


\begin{frame}{where do you get an initial guess $u^{(0)}$?}

\begin{itemize}
\item \emph{for floating ice}, assume a uniform strain rate:
   $$u^{(0)}(x) = u_g + \gamma (x-x_g)$$
\item \emph{for grounded ice}, assume ice is held by basal resistance only:
   $$u^{(0)}(x) = \left(-C^{-1} \rho g H h_x\right)^{1/m}$$
\end{itemize}
\end{frame}


\begin{frame}{abstract the ``inner'' linear problem}
\begin{itemize}
\item abstract problem:
   $$\left(W(x)\, u_x\right)_x - \alpha(x)\, u = \beta(x)$$

\vspace{-2mm}
  \begin{itemize}
  \item[$\circ$] on $0 < x < L$
  \item[$\circ$] at grounding line $x_g=0$: \quad $u(0) = u_g$
  \item[$\circ$] at calving front $x_c=L$: \quad\,\, $u_x(L) = \gamma$
  \end{itemize}
\item an \emph{elliptic} boundary value problem
\item $W(x)$, $\alpha(x)$, $\beta(x)$ are known functions in the SSA context:
  \begin{itemize}
  \item[$\circ$] both $W(x)$ and $\alpha(x)$ come from previous iteration
  \item[$\circ$] $\beta(x)$ is driving stress
  \end{itemize}
\end{itemize}
\end{frame}


\begin{frame}{numerics of the ``inner'' linear problem}

\begin{itemize}
\item suppose $j=1,2,\dots,J+1$, where $x_1 = x_g$ and $x_{J+1} = x_c$ are endpoints
\item $W(x)$ is needed on the staggered grid; the approximation is:
$$\frac{W_{j+1/2} (u_{j+1} - u_j) - W_{j-1/2} (u_{j} - u_{j-1})}{\Delta x^2} - \alpha_j u_j \stackrel{\ast}{=} \beta_j$$
\item left-hand boundary condition: $u_1 = u_g$ given
\item right-hand boundary condition:
  \begin{itemize}
  \item[$\circ$] introduce notional point $x_{J+2}$
  \item[$\circ$] approximate ``$u_x(L)=\gamma$'' by
    $$\frac{u_{J+2} - u_J}{2 \Delta x} = \gamma$$
  \item[$\circ$] use $j=J+1$ case of $\ast$ to eliminate $u_{J+2}$ (by-hand)
  \end{itemize}
\end{itemize}
\end{frame}


\begin{frame}{numerics of the ``inner'' linear problem 2}

\footnotesize
\begin{itemize}
\item thus iterate of SSA stress balance has form  \quad $A \mathbf{x} = \mathbf{b}$:
$$
\begin{bmatrix}
1 &  &  &  &  \\
W_{3/2} & A_{22} & W_{5/2} &  &  \\
 & W_{5/2} & A_{33} &  &  \\
 &  & \ddots & \ddots &  \\
 &  & W_{J-1/2} & A_{JJ} & W_{J+1/2} \\
 &  &  & A_{J+1,J} & A_{J+1,J+1} \\
\end{bmatrix}\,
\begin{bmatrix}
u_1 \\ u_2 \\ u_3 \\ \vdots \\ u_J \\ u_{J+1}
\end{bmatrix}
=
\begin{bmatrix}
u_g \\ \beta_2 \Delta x^2 \\ \beta_3 \Delta x^2 \\ \vdots \\ \beta_J \Delta x^2 \\ b_{J+1}
\end{bmatrix}
$$
\item diagonal entries
$$A_{22} = -(W_{3/2}+W_{5/2}+\alpha_1 \Delta x^2), \quad A_{33} = -(W_{5/2}+W_{7/2}+\alpha_2 \Delta x^2), \quad \dots$$
\item special cases in final equation:
$$A_{J+1,J} = 2 W_{J+1/2}$$
$$A_{J+1,J+1} = -(2 W_{J+1/2}+\alpha_{J+1}\Delta x^2)$$
$$b_{J+1} = -2 \gamma \Delta x W_{J+3/2} + \beta_{J+1} \Delta x^2$$
\item this is a \emph{tridiagonal} system
\end{itemize}
\end{frame}


\begin{frame}{numerics of the ``inner'' linear problem 3}
\label{slide:flowlinecode}

\minput{flowline}
\end{frame}


\begin{frame}{testing the abstracted linear code}

\begin{itemize}
\item first we test the linear code \texttt{flowline.m} for the abstract problem
\item test by ``manufacturing'' solutions
  \begin{itemize}
  \item[$\circ$] see \texttt{testflowline.m}
  \item[$\circ$] result: converges at optimal rate $O(\Delta x^2)$
  \end{itemize}

\bigskip
\item then write Picard iteration loop to create SSA nonlinear solver
  \begin{itemize}
  \item[$\circ$] \texttt{ssaflowline.m} on next slide
  \end{itemize}
\end{itemize}
\end{frame}


\begin{frame}{numerical SSA implementation}

\minputtiny{ssaflowline}
\end{frame}


\begin{frame}{numerical convergence}

convergence analysis of \texttt{ssaflowline.m}:

\begin{center}
  \includegraphics[width=0.75\textwidth]{shelfconv}
\end{center}
\end{frame}


\begin{frame}{SSA numerical model output}

\begin{center}
  \includegraphics[width=0.45\textwidth]{steadyshelfprofile} \quad
  \includegraphics[width=0.45\textwidth]{steadyshelfvelocity}
\end{center}

\bigskip

\begin{itemize}
\item \emph{this looks suspiciously like figures for the exact solution \dots}
\item yes
\end{itemize}
\end{frame}


\begin{frame}{numerical ice shelf modeling in 2D}

\begin{itemize}
\item ``diagnostic'' (static geometry) ice shelf modeling in 2D has been quite successful
\item observed surface velocities validate SSA stress balance model
  \begin{itemize}
  \item[$\circ$] e.g.~Ross ice shelf example below using PISM
  \item[$\circ$] \dots many models can do this
  \end{itemize}
\end{itemize}

\begin{center}
\includegraphics[width=0.45\textwidth]{rossquiver}  \hfill  \includegraphics[width=0.45\textwidth]{rossscatter}
\end{center}
\end{frame}


\begin{frame}{numerical solution of stress balances: a summary}

\begin{itemize}
\item stress balance equations (e.g.~SSA or Stokes) determine velocity from geometry and boundary conditions
\item for ice the equations are nonlinear so iteration is necessary
  \begin{itemize}
  \item[$\circ$] at each iteration a sparse matrix ``inner'' problem is solved
  \item[$\circ$] give it to a linear algebra package; don't write it yourself
  \end{itemize}
\item when geometry and boundary stresses are \emph{actually known}, many modern stress balance solvers do well
  \begin{itemize}
  \item[$\circ$] including ice shelves
  \item[$\circ$] model shallowness often is \emph{not} the issue
  \item[$\circ$] see: Aschwanden et al.~(2016), \emph{Complex Greenland outlet glacier flow captured}
  \end{itemize}
\end{itemize}
\end{frame}


\section{free advice}

\begin{frame}{best practices for numerical modeling}

\begin{itemize}
\item learn a version-control system \hfill \alert{\texttt{git}}
\item put your project on a public host from the beginning \hfill \alert{\texttt{github.com}}
  \begin{itemize}
  \item[$\circ$] ignore your secretive advisor
  \end{itemize}
\item modularize your codes
\item test the parts
  \begin{itemize}
  \item[$\circ$] make testing easily repeatable \hfill \alert{\texttt{make test}}
  \end{itemize}
\item if you are stuck:
  \begin{itemize}
  \item[$\circ$] write documentation for what you have \hfill \alert{\texttt{pdflatex}}
  \item[$\circ$] go find exact solutions and literature about submodels \hfill \alert{\texttt{google}}
  \end{itemize}
\end{itemize}
\end{frame}


\begin{frame}{}
\begin{itemize}
\item this thing

\vspace{-10mm}
\begin{center}
\includegraphics[width=0.5\textwidth]{computer}
\end{center}
\alert{does not know what you \emph{intend}}, namely your science/modelling goals

\bigskip
\item it does ``know'' your discrete equations, the program you wrote
\item you \emph{can} aspire for your computer to ``know'' your continuum model, the one that you write as PDEs
   \begin{itemize}
   \item[$\circ$] achieve this through verification with exact solutions
   \end{itemize}
\end{itemize}

\end{frame}



\begin{comment}
THINK ABOUT THESE THINGS:
* attempt to use Green's function to understand diffusivity of 1D SSA?
* why is ice sheet modeling a "1 bar subject"?
* make sure to test codes in Matlab
* props
   -- bicycle for plastic till analogy
   -- membrane and grid for free boundary
   -- silly putty
\end{comment}

\end{document}
